% 
\documentclass[10pt,letterpaper]{article}

% document setup
\usepackage[top=0.75in,bottom=0.75in,left=0.75in,right=0.75in]{geometry}
\usepackage[utf8]{inputenc}

% useful shortcuts (see mymacros.sty)
\usepackage{mymacros}

% citation style
\usepackage[backend=biber,
            style=numeric,
            doi=false,
            isbn=false,
            url=false,
            eprint=false,
            maxbibnames=99,
            sorting=none]{biblatex}
\addbibresource{refs.bib}
\renewbibmacro{in:}{}

% helvetica is a nice font
\usepackage{helvet}
\renewcommand{\familydefault}{\sfdefault}
\usepackage[letterspace=5]{microtype}

% line numbers and spacing
\usepackage[left]{lineno}
\usepackage{parskip}
\usepackage{setspace} 
\onehalfspacing

% for highlighting
\usepackage{soul,xcolor}
\definecolor{lightyellow}{rgb}{1,1,0.7}
\newcommand{\tcr}[1]{\textcolor{red}{[#1]}}
\newcommand{\tcg}[1]{\textcolor{darkgray}{\textbf{#1}}}

% for generating nonsense text to get a sense of the latex style.
\usepackage{blindtext}

% Bold the 'Figure #' in the caption and separate it from the title/caption with a period
% Captions will be left justified
\usepackage[aboveskip=1pt,
            labelfont=bf,
            labelsep=period,
            justification=raggedright,
            singlelinecheck=off,
            font=small]{caption}
\renewcommand{\figurename}{Fig}

% Leave date blank
\date{}

\begin{document}
\vspace*{0.2in}

\begin{flushleft}
{\Large
\textbf\newline{A Longer and More Formal Title For the Project}
}
\newline
\\
Author Name\textsuperscript{1, 2},
Author Name\textsuperscript{3, 4}
\\
{
\singlespacing
\small
\textsuperscript{\bf 1}First Department,
\textsuperscript{\bf 2}Second Department,
University Name, City, State, Country.\\
\medskip
\textsuperscript{\bf 3}First Department,
\textsuperscript{\bf 4}Second Department,
University Name, City, State, Country.\\
}

\end{flushleft}

\linenumbers
\lsstyle

% Abstract below 300 words
\section*{Abstract}

\blindtext

\section*{Introduction}

\Blindtext 

\section*{Results}

\subsection*{First subsection}

\blindtext

\subsection*{Second subsection}

\blindtext

\section*{Discussion}

\Blindtext 

\printbibliography

\end{document}
% 
